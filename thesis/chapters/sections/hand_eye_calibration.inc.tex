\section{Eye in hand Calibration}

Eye-in-Hand calibration is a crucial process in robotics where the spatial relationship between a robot’s end-effector (hand) and a vision sensor (eye) is determined. This calibration ensures accurate perception and interaction with the environment, allowing the robot to perform tasks with high precision.

\subsection{Tools and Software}
The MoveIt! library is widely used for motion planning and manipulation in robotics. This chapter follows the MoveIt! hand-eye calibration tutorial found at the following GitHub repository: \url{https://github.com/JStech/moveit_tutorials/blob/new-calibration-tutorial/doc/hand_eye_calibration/hand_eye_calibration_tutorial.rst}.

\subsection{Prerequisites}
Before starting the hand-eye calibration process, ensure that the following prerequisites are met:
\begin{enumerate}
    \item \textbf{MoveIt! Setup}: Ensure MoveIt! is installed and properly configured with your robot.
    \item \textbf{Aruco Marker Board}: A physical Aruco marker board must be available and detectable by the vision sensor.
    \item \textbf{Robot Control}: Ensure you have control over the robot and can move the end-effector to various positions.
\end{enumerate}

\subsection{Aruco Marker Board}
An Aruco marker board is a grid of binary square fiducial markers used for pose estimation. Each marker has a unique ID, allowing the software to determine the board's orientation and position in space.

\subsection{Calibration Procedure}

\subsubsection{Setup the Environment}
Start by launching the MoveIt! setup for your robot. Ensure the vision sensor is correctly mounted and the Aruco marker board is within the sensor’s field of view.

\subsubsection{Capture Calibration Poses}
Move the robot’s end-effector to various positions and orientations, ensuring the Aruco marker board is visible in each pose. Capture multiple poses to improve calibration accuracy. Use the following command to start the calibration capture process:
\begin{verbatim}
roslaunch moveit_calibration_gui moveit_calibration_gui.launch
\end{verbatim}

\subsubsection{Collect Data}
In the MoveIt! Calibration GUI, select the hand-eye calibration plugin and start collecting data. For each pose:
\begin{itemize}
    \item Position the robot so the Aruco board is visible.
    \item Capture the pose by clicking the "Capture" button in the GUI.
    \item Repeat this process for multiple poses (at least 10-15).
\end{itemize}

\subsubsection{Calibration Computation}
Once enough data is collected, initiate the calibration computation. The software will use the captured poses to calculate the transformation matrix between the robot’s end-effector and the vision sensor.

\subsection{Mathematical Formulation}
The hand-eye calibration problem can be described mathematically. Let \( A_i \) represent the transformation from the robot base to the end-effector and \( B_i \) represent the transformation from the camera to the marker board for each pose \( i \). The goal is to find the transformation \( X \) (from the end-effector to the camera) and \( Y \) (from the robot base to the marker board), such that:

\begin{equation}
A_i X = Y B_i
\end{equation}

This equation can be rewritten as a homogeneous transformation equation:

\begin{equation}
A_i X = Y B_i => X = A_i^{-1} Y B_i
\end{equation}

By solving this equation for multiple poses, we can determine the unknown transformations \( X \) and \( Y \).

\subsubsection{Solving the Calibration}
The software uses optimization techniques to minimize the error between the observed and predicted transformations. This is typically done using least squares fitting or other numerical optimization methods.

\subsection{Verification}
After calibration, verify the results by positioning the robot’s end-effector in known poses and checking the accuracy of the detected Aruco board position. Fine-tune the calibration if necessary.

\subsection{Conclusion}
Hand-eye calibration is a vital step in ensuring accurate robotic operations involving vision systems. By following the steps outlined in this chapter and using the MoveIt! tools, precise calibration can be achieved, leading to improved performance in robotic tasks.

This chapter covered the practical steps for hand-eye calibration using MoveIt!, described the role of Aruco marker boards, and explained the underlying mathematical formulation used by the software. Proper calibration enables robots to interact effectively with their environment, enhancing their operational capabilities.