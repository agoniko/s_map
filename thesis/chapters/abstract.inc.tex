\chapter*{Abstract}

\paragraph{What is the abstract?}

The abstract is an important component of your thesis. It is likely the first substantive description of your work read by an external examiner. You should view it as an opportunity to set accurate expectations.
The abstract is a summary of the whole thesis. It presents all the major elements of your work in a highly condensed form.
It is not merely an introduction in the sense of a preface or advance organizer that prepares the reader for the thesis. It must be capable of substituting for the whole thesis when there is insufficient time and space for the full text.

\paragraph{Size and Structure.}

Typical lenghts are within 200 and 300 words (without sections).
An abstract has to live as a stand-alone text.
The structure should mirror the structure of the whole thesis, and should represent all its major elements.
For example, if the thesis has five chapters (introduction, literature review, methodology, results, conclusion), there should be one or more sentences assigned to summarize each chapter.

\paragraph{Clearly Specify Your Research Questions.}

As in the thesis itself, your research questions are critical in ensuring that the abstract is coherent and logically structured. They form the skeleton to which other elements adhere.
They should be presented near the beginning of the abstract.

\paragraph{Don't Forget the Results.}

The most common error in abstracts is failure to present results.
The primary function of your thesis (and by extension your abstract) is not to tell readers what you did, it is to tell them what you discovered. Other information, such as the account of your research methods, is needed mainly to back the claims you make about your results.
Approximately the last half of the abstract should be dedicated to summarizing and interpreting your results.

\thispagestyle{empty}
\mbox{}
\newpage
