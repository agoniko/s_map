%-------------------------------------------------------------------------------
\chapter{Introduction}
\label{ch:intro}
%-------------------------------------------------------------------------------

This is the introduction. Here, you need to introduce your reader to the topic and guide through some logical steps to the problem formulation. Please remember who is your reader. Keep in mind that your thesis is the main deliverable of your exam work. Of course, a significant contribution (either theoretical, or scientific, or engineering) must be done in order to provide the content for the thesis. 
%------------------------------------------------------------
\section{Problem Formulation}
\label{sec:problem}
%------------------------------------------------------------

At some point, after some general introduction, you need to formulate and frame your problem. 
The problem or the goal of the project must be clear and specific. You might want to split it us into sub-goals and objectives.

%------------------------------------------------------------
\section{Outline}
\label{sec:outline}
%------------------------------------------------------------

In the end of the introduction, it is customary to provide the structure of the remaining parts of the thesis. Here is an example.

The rest of this thesis is organised as follows:

\begin{description}

  \item[\textbf{Chapter~\ref{ch:usage}}] explains how to use this template;

  \item[\textbf{Chapter~\ref{ch:guidelines}}] gives general guidelines on how to write the thesis;

  \item[\textbf{Chapter~\ref{ch:conclusions}}] contains conclusions.  Not many,
  but some.

\end{description}

Of course, you are free to structure the outline in a different form.
