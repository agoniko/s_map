%
\chapter{Using This Template}
\label{ch:usage}

Writing in \LaTeX is not complicated, but you need to understand how to use the different commands \cite{latex2e}. Here we explain those specific to this template.

To start using this template, your \texttt{.tex} file should start with the lines \\[3mm]
\verb|\documentclass[bac,screen]{bai-thesis}| \\
\verb|\usepackage{thesis}|

This initializes the template and imports the main commands. The rest depends on your content.

\section{Front Matter}

The title and author are set with the commands %\\[3mm]
\verb|\title{}| and 
\verb|\author{}|

If you need to put a date, use the command \verb|\date{}|. \verb|\today| returns always the current date; you can use it to keep the date updated automatically every time you compile your document.

The last three things to specify are the subject, academic year, and name of the supervisor(s):
\begin{itemize}
    \item \verb|\subject{Artificial Intelligence}|
    \item \verb|\AY{2023/2024}|
    \item \verb|\supervisor{First Supervisor, Affiliation}|
\end{itemize}
%
If the thesis was supervised by more than one person, simply use the \verb|\supervisor{}| command several times: \\[3mm]
\verb|\supervisor{First Supervisor, Affiliation}| \\ 
\verb|\supervisor{Second Supervisor, Affiliation}|

The description of the front matter finishes with 
the two commands \\[3mm]
\verb|\begin{document}| \\
\verb|\maketitle| \\[3mm]
to start the document and create the title page.

\section{Main Matter}

The main content of your thesis starts then. You should first
provide your abstract and keywords with, unsurprisingly, the 
\texttt{abstract} environment and the \texttt{keywords} command. 
A simple example is the following
\begin{verbatim}
    \begin{abstract}
      here is the abstract
    \end{abstract}
    \keywords{Key, Word, Another}
\end{verbatim}

This should be followed by the table of contents and, if needed,
tables of figures, tables, and algorithms. For the moment, modify
the file \texttt{contents.tex} to fit your needs.

You then present all the important information. You must cite your
sources. You can cite works by using their keys in the bib files, like in this example~\cite{latex2e,LiVR22,ChenZH23}.
%
The bib file can be created by hand or imported from other applications, such as Mendeley or CrossRef.

Hyperrefs are enabled.  This means references (to chapters, sections, figures, tables, algorithms, etc) are links, which can be clicked to navigate through the thesis.  Also, backrefs are enabled, which means that references include ``(Cited on page(s))'' links (see the examples in the references). You can also introduce citations that are never mentioned in
the text (although this should only happen in very extreme situations). In that case, the backref will explicitly mention that the reference is not cited.

\bigskip

You will likely introduce figures, tables, and algorithms. There are
standard environments to do this. Make sure that they are ``floating''
(at the top or at the bottom of the page, not between the text) and that
they are all references in the main text.

Figure~\ref{fig:exa} contains a picture generated by Dall-E. Please note the realistic position of the typewriter and of the pen at the person's hand. The caption should briefly explain the figure, and should always appear below it.  
%
\begin{figure}[tb]
\centering
\includegraphics[width=0.75\textwidth]{figures/typewriter} 
\caption[Short caption for list of figures]{Long caption. This figure was automatically generated by Dall-E to represent the happy and productive process of writing a thesis.}
\label{fig:exa}
\end{figure}

Table~\ref{tab:exa} contains some made up stuff. Tables can be
very complicated. The template imports the \texttt{booktabs} package
which allows for some better looking tables. To make cleaner and more
elegant tables, avoid vertical lines as much as possible. For tables,
specially if they are long, the caption should appear at the top.
%
\begin{table}[tb]
\centering
\caption
[Short caption for list of tables.]
{Long caption.  This is a table which uses some advanced features like
\texttt{multirow} and \texttt{multicolumn} and row separators}
\label{tab:exa}
\begin{tabular}{cccc}
\toprule
\multicolumn{2}{c}{\textbf{Categories}} &
  \textbf{Heading1} &
  \textbf{Heading2} \\
\midrule
\multirow{2}{*}{\textbf{Heading3}} &
  \textbf{Heading5} &
      Data element
     &
      Data element
     \\
  \cmidrule{2-4}
&
  \textbf{Heading6} &
      Some stuff in a table.
     &
      Some stuff in a table.
     \\
\midrule
Some & Other & Data & Here \\
\bottomrule
\end{tabular}
\end{table}
%
If the table is very long, you may want to use a coloring scheme on
some rows to help reading (and reduce the use of horizontal lines).

Finally, we also present an example of an algorithm in Algorithm~\ref{alg:exa}.
%
This is constructed using the \texttt{algorithmic} package, but you are free to choose other packages according to your needs. The caption is also at the top, but it is usually short. The workings of the algorithm are described in the main text.
%
\begin{algorithm}[tb]
\begin{algorithmic}[1]
\REQUIRE The first input.
\REQUIRE The second input.
\ENSURE The output.
\medskip
\STATE First statement.
\FORALL{Something.}
  \IF{A condition.}
    \STATE Statement.
  \ELSIF{Another condition.}
    \STATE Statement.
  \ELSE
    \STATE Statement.
  \ENDIF
\STATE Statement.
\ENDFOR
\end{algorithmic}
\caption[Short caption for list of algorithms.]
{Long caption.  This is an algorithm.}
\label{alg:exa}
\end{algorithm}

\section{Back Matter}

After you have written all your text (including your conclusions) you
still need to add the reference (bibtex) file and call the command to
create the reference section. Modify the file \texttt{backmatter.tex}
to achieve this. The additional (seemingly obscure) commands used there
help building an adequate table of contents, and allow for pdf readers
to get a navigation index.
